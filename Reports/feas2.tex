\documentclass[12pt,letterpaper,twoside]{article}
\usepackage{amsmath}
\usepackage{amsfonts}
\usepackage{appendix}
\usepackage[figurewithin=section,tablewithin=section]{caption}
\usepackage[usenames,dvipsnames]{color}
\usepackage{graphicx}
\usepackage{longtable}
\usepackage{rotating}
%\usepackage{verbatim}
\usepackage[pdftex,bookmarksopen=false]{hyperref}
%\usepackage[pdftex]{hyperref}
\hypersetup{pdfauthor={John Sibert}}
\hypersetup{pdfsubject={Assessment Model of MHI YFT Stocks}}
\hypersetup{pdftitle={Assessment Models for Hawaiian Islands
Yellowfin Tuna Fishery}}
\hypersetup{pdfkeywords={yellowfin,state space, biomass transfer, model,Hawaii}}

\newcommand\doublespacing{\baselineskip=1.6\normalbaselineskip}
\newcommand\singlespacing{\baselineskip=1.0\normalbaselineskip}
\renewcommand\deg[1]{$^\circ$#1}
\newcommand\SD{SEAPODYM}
\newcommand\MFCL{MULTIFAN-CL}
\newcommand\ADMB{ADModel Builder}
\newcommand\SPC{Secretariat of the Pacific Community}
\newcommand\WCPO{Western Central Pacific Ocean}
\newcommand\SSAP{Skipjack Survey and Assessment Programme}
\newcommand\RTTP{Regional Tuna Tagging Programme}
\newcommand\PTTP{Pacific Tuna Tagging Programme}
\newcommand\FAD{fish aggregating device}
\newcommand\ADRM{advection-diffusion-reaction model}
\newcommand\help[1]{\color{Magenta}{\it #1 }\normalcolor}
\newcommand\widebar[1]{\overline{#1}}
\newcommand\EEZ{Exclusive Economic Zone}

\newcommand\None{{N_{1,1}}}
\newcommand\Ntwo{{N_{2,1}}}
\newcommand\Nsum{{N_{1,1}+N_{2,1}}}
\newcommand\peryr{yr$^{-1}$}
\newcommand\prevN[1]{{#1_{t-\Delta t}}}
\newcommand\nextN[1]{{#1_t}}
\newcommand\MSY{\tilde{Y}}
\newcommand\Fmsy{F_{\MSY}}

\title{Feasibility of developing a stock assessment model for Main
Hawaiian Islands Yellowfin Tuna Fishery\\
\vspace{2ex}
Part Deux}

\author{
John Sibert\thanks{sibert@hawaii.edu}\\
Joint Institute of Marine and Atmospheric Research\\
University of Hawai'i at Manoa\\
Honolulu, HI  96822 U.S.A.\\[0.125in]
\date{\today}
}

\pagestyle{myheadings}
\markboth{John Sibert\hfil MHI Yellowfin Assessment Model
Feasibility --- 2}
{MHI Yellowfin Assessment Model Feasibility --- 2\hfil John Sibert}

\begin{document}
% amsmath package
\numberwithin{equation}{section}
\numberwithin{figure}{section}
\maketitle

%\doublespacing


\section{Executive Summary}
A previous evaluation of the feasibility of developing a
stock assessment model for the Main Hawaiian Islands yellowfin tuna
fishery concluded that such a model is feasible, but that estimation of
fishery management reference points is difficult (Sibert, 2015).
The ambiguity of this
conclusion is unsatisfying, and further research was undertaken to
develop more practical alnternative models. 
The 2015 model attempted to represent
the dynamics of two coupled tuna populations with immigration,
emigration, and fishing and required estimation of approximately 15
model parameters.

Single population models with simplified model
structure
were examined. The simplified models are capable of
estimating critical reference points such as maximum sustainable
yield (MSY) and fishing mortality at MSY. 
%Estimates of MSY range between 1100mt and 1300mt per year (Table~\ref{tag:ests4}).
Estimated biomass trends between 1952 and 2012 
(Figure~\ref{fig:estbiomass}) differ in general level, but
qualitatively consistent across models. All models estimate a period of high
biomass between 1970 and 1990, 
including periods where the biomass exceeds the
equilibrium biomass or ``carrying capacity''. 
These models are
promising staring places to begin development of a full-featured stock
assessment model for the MHI yellowfin fishery.

\section{Methods}
Model feasibility depends equally on both the model and on whether the
available data
contain sufficient information to estimate model parameters.
Evaluation of feasibility challenges both the model and the available
data. The data used in this analysis are the annual catch data from
five gear types fishing within 200 nm of the Main Hawaiian Islands
(Sibert 2015).

The basic model is  state-space implementation of the well known
Schaefer biomass dynamics model (Schaefer, 1954).
The principle model assumptions are:
\begin{enumerate}
\item The dynamics of the population of YFT in the MHI follows a
simple Schaefer model with MHI-specific growth parameters.
\item Fishing mortality is represented by a random walk as in the
2015 two-population model (Nielsen and Berg, 2014).
\item The local dynamics are ``forced'' by assuming that the local
abundance is approximately proportional to some ``index'' population
such as the abundance of the larger Pacific population estimated by
another model.
\end{enumerate}

The basic model was implemented with two different parameter
substitutions in an attempt to find parameter combinations that are
more easily estimated. Indexing was implemented flexibly to
demonstrate the effects of omitting the index altogether.

Model estimates of yellowfin biomass by \MFCL\ (MFCL)
from the most recent WCPFC stock assessment (Davies et al 2014) provide
a convenient choice of biomass indices. The MHI straddle the
boundary between MFCL regions 2 and 4. Region 2 is arguably more similar
ecologically to the MHI than the more equatorial region 4, and the population
in region 4 is much larger and exhibits a greater level of depletion than the
population in region 2. Estimated biomass trends from MFCL region 2
were used as the abundance index for most model feasibility evaluations.

Complete details of model structure are presented in Appendix~\ref{sec:models}.



\section{Results}

\begin{table}
\caption{
Model estimates from four different model configurations.
Model complexity, expressed in number of parameters estimated ($n$)
increases from left to right. Long dashes (---) indicate parameters
not estimated. 
{%\small
%$n$ indicates the number of parameters estimated; 
$-\log L$ is the negative log likelihood (the smaller the number, the
better the fit);
$|G|_{max}$ is the curvature of the likelihood at the its apparent
minimum (values greater than $0.001$ indicate non-convergence);
other variables are defined in Table~\ref{tab:allvars1}.
}
\label{tag:ests4}}
\begin{center}
\begin{tabular}{|l|rr|rr|}
\hline
Forcing & \multicolumn{2}{c|}{None}&\multicolumn{2}{c|}{MFCL 2}\\
\cline{2-3}\cline{4-5}
Parameterization&$\MSY\quad\Fmsy$&$B_1\quad d$&$\MSY\quad\Fmsy$&$B_1\quad d$\\
Designation& A & B& C& D\\
\hline
\hline
$n$ & 4 & 5 & 5 & 6\\
$-\log L$ & -237.238 & -237.221 & -247.175 & -242.816\\
$|G|_{max}$ & 0.0016409 & 2.11959e-06 & 3.51082e-05 & 29.6554\\
$B_1$ & --- & 5043.2 & --- & ---\\
$d$ & --- & 1.3675 & --- & ---\\
$\MSY$ & 1147.5 & 1083 & 1288.7 & ---\\
$\Fmsy$ & 0.82239 & 0.31408 & 0.1668 & ---\\
$r$ & 1.6448 & 0.62815 & 0.3336 & ---\\
$K$ & 2790.8 & 6896.6 & 15452 & ---\\
$\sigma_P$ & 0.37416 & 0.37042 & 0.2743 & ---\\
$\sigma_Y$ & 0.41693 & 0.41937 & 0.46924 & ---\\
$Q$ & --- & --- & 0.04321 & ---\\
\hline
\end{tabular}
\end{center}
\end{table}

Fitting results are summarized in Table~\ref{tag:ests4}.
Both non-indexed models (A and B) converged to well-defined solutions.
Only the $\MSY\;\Fmsy$ indexed parameterization (model C) converged. 
(The value of $|G|_{max}$ for the $B_1\; d$ parameterization ,model D,
was much higher than the value indicating convergence to a
well-defined solution.)
Estimated values of $r$ and $K$ and corresponding estimates of $\Fmsy$
and $\MSY$ depend heavily on the use of the biomass index. The
indexed model estimates a larger and slower growing stock than the
non-indexed models.

\begin{figure}
\begin{center}
\includegraphics[width=\textwidth]{./graphics/biomass-array.pdf}
\caption{\label{fig:estbiomass}
Estimated biomass trends for three model configurations, indicated in
the upper left of each panel.
Heavy blue lines indicate the estimated biomass in mt.
The purple line indicates the estimated biomass index.
The light blue shaded areas represent the process error as 
$\pm 2\sigma_P$.
The equilibrium biomass $K$ is indicated by the blue dot-dash line.
}
\end{center}
\end{figure}

Estimated biomass trends for the 3 converged models are shown in
Figure~\ref{fig:estbiomass}.
A five-fold difference in estimated population size is the
most striking difference between the 3 models reflecting similar
differences in estimates of $K$.
Relatively high stock sizes in the late 1970s and early 1980s is
apparent in all three models. 
This period of abnormally high biomass is most obvious in the indexed model
which reflects the high biomass in MFCL region 2 during that period.
It is encouraging to
detect this period of high biomass in the two non-indexed models as well.

\begin{figure}
\begin{center}
\includegraphics[width=\textwidth]{./graphics/production-array.pdf}
\caption{\label{fig:estprod}
Catch plotted against fishing mortality (production curves)
for three model configurations, 
indicated in the upper left of each panel.
The heavy green line and dark green dots are estimated catch plotted
against.
The green $+$ symbols are the observed catch annotated with the year.
The dashed red line is the theoretical equilibrium yield.
Note that the scale of the abscissa is different in each panel.
}
\end{center}
\end{figure}

Production curves for the 3 converged models are shown in
Figure~\ref{fig:estprod}.
In the two non-indexed models, the peaks of the production curves
(dashed red line) are close to the highest estimated fishing mortality
and near the highest observed catches. 
In contrast, the peak of the production curve for the indexed model is
far the right of the highest estimated fishing mortality.
Catches exceed the production curve between the mid 1970 through the
mid 1990s and drop below the production curve after 2000 for all three
models.





\section{Discussion}


\vspace{4ex}
%\clearpage
\noindent {\bf Acknowledgements.}
This work was funded by the Western Pacific Regional Fisheries
Managment Council. I thank the Council for its generous support and
Council Staff Paul Dalzell and Eric Kingma for encouraging me to
actually take on this challenging project and for their on-going
collaboration.
Thanks to Dr. David Fournier for suggesting the ``$B_1\;d$''
parameterization of the Schaefer model.
to Mr. David Itano for sharing insights into the small-boat
fisheries in Hawaii,
to Mr. Reginald Kokubun of the Hawaii Division of Aquatic
Resources for supplying catch report data from the HDAR commercial
fisheries data base,
to Mr. Keith Bigelow and Ms. Karen Sender of NOAA Pacific
Island Fisheries Science Center for supplying logbook reporting data and
weight-frequency data from the PIFSC data base.
Thanks also to Dr. John Hampton of the Secretariat of the Pacific
Community, Oceanic Fisheries Programme, for making available
MULTICAN-CL output files from the latest Western and Central Pacific
Fisheries Commission yellowfin tuna stock assessment, and to Mr. Nick
Davies for sharing R scripts and advice on interpreting \MFCL\ output files.

\section*{References}
{\parindent=0cm \small
\everypar={\hangindent=2em \hangafter=1}\par
%\doublespacing
Carruthers, T. and M. McAllister. 2011.
Computing prior probability distributions for the
intrinsic rate of increase for atlantic tuna and
billfish using demographic methods.
Collect. Vol. Sci. Pap. ICCAT, 66(5): 2202-2205.

Davies, N., S. Harley, J. Hampton, S. McKechnie. 2014. Stock
assessment of yellowfin tuna in the western and central pacific ocean.
WCPFC-SC10-2014/SA-WP-04.

Fournier, D. A., H.J. Skaug, J. Ancheta, J.Sibert, J. Ianelli, 
A. Magnusson, M. N. Maunder, A. Nielsen. 2012. AD Model Builder:
using automatic differentiation for forstatistical inference of highly
parameterized complex nonlinear models. Opti-mization Methods and
Software 27, 233–249.

Nielsen, A., C. Berg. 2014. Estimation of time-varying selectivity
in stock assessments using state-space models. Fisheries Research
158:96-101.

Murray, J. D. 1993. Mathematical biology. Second Edition.
Springer-Verlag. 767pp.

Quinn, T, R. Deriso. 1999. Quantitative fish dynamics. Oxford
University Press, New York.

Schaefer, M. B. 1954. Some aspects of the dynamics of populations
important to the management of the commercial marine fisheries. IATTC
Bull. 1(2):27-56.

Sibert, J. R. 2015. Feasibility of devloping a stock assessment
model for Main Hawaiian Islands Yellowfin Tuna Fishery.
119th Meeting of the Scientific and Statistical Committee.
Document 7.A.1(1)Rev 1.

Skaug, H., Fournier, D., 2006. Automatic approximation of the marginal
likelihood in non-Gaussian hierarchical models. Computational
Statistics \& Data Analysis 51, 699–709.

Wells, D., J. Rooker, D. Itano. 2012.  Nursery origin of yellowfin
tuna in the Hawaiian Islands. Mar. Ecol. Prog. Ser. 461:187-196. 
\par}

\clearpage

appendix
\section{Model development}
\label{sec:models}
State-space models separate variability in the biological
processes in the system (transition model)
from errors in observing features of interest
in the system (observation model), g.g. Berg and Nielsen (2014). 

{\bf Transition Model $T(\alpha_{t-1})$.}
The general form of the transition model is
\begin{equation}
\alpha_t=T(\alpha_{t-1}) + \eta_t
\end{equation}
where $\alpha_t$ is the state at time $t$ and 
the function $T$ embodies the dynamics mediating the
development of the state at time $t$ from the state at the previous
time with random process error, $\eta_t$.

Stock dynamics of follow the classic Schaefer form:
\begin{equation}
\label{eqn:ischaefer}
\frac{dN}{dt} = rN(1-\frac{N}{K}) - FN
\end{equation}
where $N$ is the biomass of YFT in the MHI, 
$r$ is the logistic growth rate per year,
$K$ is the asymptotic biomass, and
$F$ is the total fishing mortality per year in the MHI.

The state space transition equation for the single population model is
developed by solving \ref{eqn:ischaefer} analytically from one time to
the next (see Appendix \ref{sec:schaefer}).
\begin{equation}
\label{eqn:intschaeferA}
N_t = \frac{K(r-\bar{F}_t)}{\frac{K(r-\bar{F}_t)}{N_{t-\Delta t}}e^{-\Delta
t(r-\bar{F}_t)}-re^{-\Delta t(r-\bar{F}_t)} -r} \cdot e^{\eta_t};
\quad \eta_t\sim N(0,\sigma^2_N)
\end{equation}
where 
$\bar{F}_t$ is the total fishing mortality, i. e.,
$$
\bar{F}_t =\sum_{g=1}^n F_{g,t-\Delta t}.
$$
and $\eta_t \sim N(0,\sigma_N)$ is a process error expressing
variability in population dynamics.

Fishing mortality is treated as a random walk.
The logarithm of fishing mortality is assumed to
follow a random walk with normal increments, i.e.,
\begin{equation}
\label{eqn:Fwalk}
\log F_{g,t} = \log F_{g,t-1} + \xi_t;\quad \xi_t\sim
N(0,\sigma^2_F)
\end{equation}
where  $\sigma^2_F$ is a process error expressing the year to year
variability in fishing mortality.

The indexed abundance model
assumes that the biomass of YFT in the MHI
is proportional to the biomass of an ``index'' population.
\begin{equation}
\log N_t - \log (Q\cdot I_t) + \omega_t;\quad \omega_t\sim N(0,\sigma^2_I)
\label{eqn:index}
\end{equation}
where
$I_t$ is the size of the index population at time $t$,
$Q$ is the estimated ratio of the MHI population size to the index
population,
and $\sigma^2_I$ is a process error representeing the difference
between the MHI biomass and the abundance index. 

All process errors are assumed to have the same distribution
$N(0,\sigma^2_P)$; $\sigma_N$,$\sigma_F$, and $\sigma_I$ are assumed
to be equal and estimated as a global process error with variance
$\sigma^2_P$.

The logistic parameters $r$ and $K$ are notoriously difficult to estimate
accurately. Two different
alternative parameter substitutions were tested. In the first
alternative, MSY ($\MSY$) and fishing mortality at MSY ($\Fmsy$) were
estimated directly and substituted in in \ref{eqn:intschaeferA} as
$r=2F_{\MSY}$ and $K=\frac{4\MSY}{r}$.
In the second alternative, the initial population size $B_1$ and a
proportionality constant $d$ are estimated directly and substituted in
\ref{eqn:intschaeferA} as $K=d\cdot B_1$.


{\bf Observation Model, $O(\alpha)$.}
The general form of the observation model is
\begin{equation}
x_t = O(\alpha_t) + \varepsilon_t
\end{equation}
where the function $O$ describes the measurement process with
error $\varepsilon$ in observing the population.

Predicted catch, $\widehat{C}_{g,t}$, for each gear is the product of
estimated fishing mortality and the total biomass.
\begin{equation}
\widehat{C}_{g,t} = F_{g,t}\cdot\Bigl(\frac{N_{t-\Delta
t}+N_t}{2}\Bigr) \cdot e^{\varepsilon_t}
\label{eqn:obs1}
\end{equation}
where the total biomass is  the average
biomass over the time step (Quinn and Deriso, 1999), and
$\varepsilon_t$ is a ``zero-inflated'' log normal likelihood given by
\begin{equation}
  \log \varepsilon_t = \left\{
    \begin{array}{r@{\;:\quad}l}
       C_{g,t} > 0 &
(1-p_0)\cdot\bigg(\log\frac{1}{\sqrt{2\pi\sigma^2_Y}}
          -\Bigl(\frac{\log
C_{g,t}-\log\widehat{C}_{g,t}}{\sigma_Y}\Bigr)^2\bigg)\\
       C_{g,t} = 0 & p_0 \cdot\log \frac{1}{\sqrt{2\pi\sigma^2_Y}}\\
    \end{array}
  \right.
\end{equation}
where $\sigma_Y$ is the observation error and
$p_0$ is the proportion of observed catch observations equal to zero.
This proportion may be estimated or fixed at a constant value. For
current analysis, it is fixed at $p_0 = 0.15738$ as computed from the data.
% line 186, file issams.cpp, prop_zero =  0.4918 0 0 0.098361 0.19672, tprop_zero = 0.15738



\begin{table}
\caption{Complete list of parameters for two alternative
parameterizations of the state-space single population Schaefer model.
\label{tab:allvars1}}
\begin{center}
\begin{tabular}{lll}
\hline
$\MSY\;\Fmsy$ & $B_1\;d$ & Definition\\
\hline
\hline
$\Fmsy$&  & Fishing mortality at maximum sustainable yield\\
$\MSY$ &  & Maximum sustainable yield\\
    & $r$ & Instantaneous growth rate\\
    & $B_1$ & Biomass at time of first observation\\
    &  $d $ & Constant of proportionality; $B_1=Q\cdot K$\\
$Q$ & $Q$ & Abundance index proportionality constant\\
$\sigma_P$ & $\sigma_P$ & Global process error SD; $\sigma_P=\sigma_N=\sigma_F=\sigma_Q$\\
$\sigma_Y$ & $\sigma_Y$ & Observation error SD \\
\hline
$p_0$ & $p_0$ & Proportion of zero catch observations;\\
      &        & fixed at $p_0 = 0.15738$\\
\hline
\end{tabular}
\end{center}
\end{table}

{\bf Priors} $r$ or $\Fmsy$?


{\bf Estimation.} The model states, $N_t$ and $F_{gt}$, are assumed to be random
effects (Skaug and Fournier 2006). Model parameters are estimated by
maximizing the joint likelihood of the random
effects and the observations.
\begin{equation}
L(\theta,\alpha,x)=
\prod^m_{t=2}\big[\phi\big(\alpha_t-T(\alpha_{t-1}), \Sigma_\eta\big)\big]
\prod^m_{t=1}\big[\phi\big(x_t-O(\alpha_t), \Sigma_\varepsilon\big)\big]
\end{equation}
Here, $m$ is the number of time steps in the catch time series and
$\theta$ is a vector of model parameters.
The model is implemented in ADMB-RE (Fournier et al 2012).
The actual number of
parameters to be estimated depends on the model configuration,
specified by phase flags in the input file. 
All computer code, data files, and draft reports in support of this
analysis can be found at Github:
\url{https://github.com/johnrsibert/XSSA.git}.



\clearpage
\section{Integrating Schaefer Models}
\label{sec:schaefer}
The widely used Schaefer (1954) fisheries stock assessment model 
is a simple
extension of the logistic population model with a term added to
represent removals from the population due to fishing.
\begin{equation}
\label{eqn:sschaefer}
\frac{dN}{dt} = rN(1-\frac{N}{K}) - FN = N(r-F-\frac{r}{K}N)
\end{equation}
where $N$ is the population size,
$r$ is the instantaneous growth rate ($t^{-1}$),
$K$ is the asymptotic population size in the same units as $N$,
and $F$, is the instantaneous rate of removal due to fishing ($t^{-1}$).
Equation (\ref{eqn:sschaefer}) reduces to the logistic model
if $F$ is assumed to be zero.
These models are usually integrated numerically with
``explict'' finite difference methods to compute an approximation 
of the value of $N$ at some time. 
Such approximations are often unstable for 
values of $r$ large relative to the time step used in the finite
difference solution.
The performance of estimation methods in which logistic models
are embedded in a statistical procedure depending on numerical function
minimizers, i.e, models build using ADMB, TMB and BUGS,
is greatly improved in accuracy, speed and use of
computing resources if analytical solutions are used in preference
to finite difference approximations. 

\subsection{Single population}
Quinn and Deriso (1999) show the integral of the logistic ODE and
point out that it can be obtained using integration by partial fractions.
Numerous mathematics tutorials can be found on the World Wide Web use
integration of the logistic ODE to illustrate the technique of
integration by partial fractions.
The same procedure can be used to integrate the Schaefer ODE.
Equation~(\ref{eqn:sschaefer}) is rearranged slightly and variables
separated to become
\begin{equation}
\frac{K}{N(K(r-F)-rN)}dN=dt.
\end{equation}
The fraction in the left hand side can be factored into two parts,
\begin{equation}
\frac{K}{N(K(r-F)-rN)}=\frac{A}{N}+\frac{B}{(K(r-F)-rN)}.
\end{equation}
$A$ and $B$ are constants that can be found by solving
$K=A(K(r-f)-rn))+BN$
setting $N=K$ and $N=0$; 
$A=\frac{1}{r-F}$ and $B=1+\frac{F}{r-F}$.
The desired integral becomes
\[\int\frac{K}{N(K(r-F)-rN)}dN   = \int dt\]
\[\int\frac{A}{N}dN + \int\frac{B}{K(r-F)-rN}dN  = \int dt\]
\[\frac{1}{r-F}\int\frac{1}{N}dN + (1+\frac{F}{r-F})\int\frac{1}{K(r-F)-rN}dN  = \int dt\] 
\[\frac{1}{r-F}\log |N| + \frac{1}{r}(1+\frac{F}{r-F})\log |K(r-f)-rN| +log C  = t\] \[\log |N| - \log |K(r-F)-rN| + log C  = t(r-F)\]
\[\frac{|N|}{|K(r-F)-rN|}\cdot C  =  e^{t(r-F)}\]
\[\frac{|K(r-F)-rN|}{C|N|} =  e^{-t(r-F)}\]
where $C$ is the constant of integration.
Setting $|N| = N_t$, the population size at time $t$, yields
\begin{equation}
\label{eqn:NtC}
N_t=\frac{K(r-F)}{Ce^{-t(r-F)}+r}
\end{equation}
A formula suitable for computing population size at successive
time steps can be found by setting $N_t = N_{t-\Delta t}$ at time
$t=t-\Delta t$ in equation (\ref{eqn:NtC}).
The integration constant, $C$, becomes
\begin{equation}
C=\Bigg(\frac{K(r-f)}{N_{t-\Delta t}}-r\Bigg)e^{(t-\Delta t)(r-F)},
\end{equation}
and finally
\begin{equation}
\label{eqn:intschaefer}
N_t = \frac{K(r-F)}{\frac{K(r-F)}{N_{t-\Delta t}}e^{-\Delta t(r-F)}-re^{-\Delta t(r-F)} -r}
\end{equation}
\help{
Further simplification of this equation may be possible, but I have
not found it. In any case, equation (\ref{eqn:intschaefer}) is the
only general solution of the Schaefer ODE that I have seen, and it appears
to work well in numerical applications.
}

\subsection{Two populations with exchange}
The motivation for the two popultion Schaefer model with exchange is
developed fully in Appendix~\ref{sec:models}.

The basic equations can be written
\begin{eqnarray}
\label{eqn:xschaefer1}
\frac{d\None}{dt}&=&\None\Big(r-F-T_{12}-2(1-q)\frac{r}{K}\Ntwo-\frac{r}{K}\None\Big)\\
\nonumber\\
\label{eqn:xschaefer2}
\frac{d\Ntwo}{dt}&=&\Ntwo\Big(r-F-T_{12}-2q\frac{r}{K}\None-\frac{r}{K}\Ntwo\Big)+T_{21}
\end{eqnarray}
where $\None$ the biomass of fish originating in region~1
and residing in region~1,
and $\Ntwo$ is the biomass of fish originating in region~2
but residing in region~1.
The parameters $r$, $K$ and $F$ are unchanged from
equation~(\ref{eqn:sschaefer}), and
$T_{12}$ is the emigration rate from region~1 ($t^{-1}$), 
$T_{21}$ is the rate of immigration of biomass from region~2 to
region~1 in units of biomass per time,
and $q;\; (0 < q < 1)$ partitions the mortality caused by ``competition''
between the two subpopulations.
Substitute
\begin{eqnarray}
\label{eqn:Zdef}
Z_1&=&F+T_{12}+2(1-q)\frac{r}{K}\Ntwo\\
\nonumber\\
Z_2&=&F+T_{12}+2q\frac{r}{K}\None
\end{eqnarray}
into equations~(\ref{eqn:xschaefer1}) and~(\ref{eqn:xschaefer2})
respectively to produce model equations in a similar form to
equation~(\ref{eqn:sschaefer})
\begin{eqnarray}
\label{eqn:xschaeferZ1}
\frac{d\None}{dt}&=&\None(r-Z_1-\frac{r}{K}\None)\\
\nonumber\\
\label{eqn:xschaeferZ2}
\frac{d\Ntwo}{dt}&=&\Ntwo(r-Z_2-\frac{r}{K}\Ntwo)+T_{21}.
\end{eqnarray}
Equation~(\ref{eqn:xschaeferZ1}) can be integrated in the same manner as
equation~(\ref{eqn:sschaefer}) to yield
\begin{equation}
\label{eqn:N1tC}
\None_t=\frac{K(r-Z_1)}{C_1e^{-t(r-Z_1)}+r}.
\end{equation}

\help{All that remains is to find an equivalent integral for
equation~(\ref{eqn:xschaeferZ2}) and a means to simultaneously solve
for $C_1$ and $C_2$. }

%\begin{equation*}
%{{2\,\log \Big|{{2\,a\,x-\sqrt{b^2-4\,a\,c}+b}\over{2\,a\,x+
% \sqrt{b^2-4\,a\,c}+b}}\Big|}\over{\sqrt{b^2-4\,a\,c}}}+C=t
%\end{equation*}
%
%\begin{equation*}
%x=-{{\sqrt{b^2-4\,a\,c}\,\left(e^{{{\sqrt{b^2-4\,a\,c}\,t
% }\over{2}}-{{\sqrt{b^2-4\,a\,c}\,C}\over{2}}}+1\right)+b\,\left(e^{
% {{\sqrt{b^2-4\,a\,c}\,t}\over{2}}-{{\sqrt{b^2-4\,a\,c}\,C}\over{2}}}
% -1\right)}\over{a\,\left(2\,e^{{{\sqrt{b^2-4\,a\,c}\,t}\over{2}}-{{
% \sqrt{b^2-4\,a\,c}\,C}\over{2}}}-2\right)}}
%\end{equation*}
%
%
\clearpage
%\input{N1}
%\input{N2}


\end{document}

\begin{figure}
\begin{center}
\begin{tabular}{lr}
\includegraphics[width=0.495\textwidth]{./graphics/biomass-array.pdf}&
\includegraphics[width=0.495\textwidth]{./graphics/production-array.pdf}\\
\end{tabular}
\end{center}
\caption{\label{fig:estprod}Blah blah}
\end{figure}
