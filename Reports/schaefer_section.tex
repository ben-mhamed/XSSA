\section{Integrating Schaefer Models}
\label{sec:schaefer}
The widely used Schaefer (1954) fisheries stock assessment model 
is a simple
extension of the logistic population model with a term added to
represent removals from the population due to fishing.
\begin{equation}
\label{eqn:sschaefer}
\frac{dN}{dt} = rN(1-\frac{N}{K}) - FN = N(r-F-\frac{r}{K}N)
\end{equation}
where $N$ is the population size,
$r$ is the instantaneous growth rate ($t^{-1}$),
$K$ is the asymptotic population size in the same units as $N$,
and $F$ is the instantaneous rate of removal due to fishing ($t^{-1}$).
Equation (\ref{eqn:sschaefer}) reduces to the logistic model
if $F$ is assumed to be zero.
Logistic models are usually integrated numerically with
``explicit'' finite difference methods to compute an approximation 
of the value of $N$ at some (future) time. 
Such approximations are often unstable for values of $r$ large
relative to the time step used in the finite difference solution.
Statistical procedure depending on numerical function
minimizers, i.e, models built using ADMB, TMB and BUGS,
do not perform well in the face of numerical instabilities in solving
differential equations.
The accuracy, speed, and use of computing resources of 
estimation methods involving logistic models are
greatly improved analytical solutions can be used in preference
to finite difference approximations. 

\subsection{Single population}
The integral of the logistic differential equation can be found in
several places;
Murray (1993), Quinn and Deriso (1999).
Numerous mathematics tutorials are available on the World Wide Web
that use
integration of the logistic differential equation as an exercise to
illustrate the technique of integration by partial fractions.
The same procedure can be applied to the Schaefer differential
equation.
Equation~(\ref{eqn:sschaefer}) is rearranged and variables
separated to become
\begin{equation}
\frac{K}{N(K(r-F)-rN)}dN=dt.
\end{equation}
The fraction in the left hand side can be factored into two parts,
\begin{equation}
\frac{K}{N(K(r-F)-rN)}=\frac{A}{N}+\frac{B}{(K(r-F)-rN)}.
\end{equation}
$A$ and $B$ are constants that can be found by solving
$K=A(K(r-f)-rn))+BN$
setting $N=K$ and $N=0$; 
$A=\frac{1}{r-F}$ and $B=1+\frac{F}{r-F}$.
The desired integral becomes
\[\int\frac{K}{N(K(r-F)-rN)}dN   = \int dt\]
\[\int\frac{A}{N}dN + \int\frac{B}{K(r-F)-rN}dN  = \int dt\]
\[\frac{1}{r-F}\int\frac{1}{N}dN + (1+\frac{F}{r-F})\int\frac{1}{K(r-F)-rN}dN  = \int dt\] 
\[\frac{1}{r-F}\log |N| + \frac{1}{r}(1+\frac{F}{r-F})\log |K(r-f)-rN| +log C  = t\] \[\log |N| - \log |K(r-F)-rN| + log C  = t(r-F)\]
\[\frac{|N|}{|K(r-F)-rN|}\cdot C  =  e^{t(r-F)}\]
\[\frac{|K(r-F)-rN|}{C|N|} =  e^{-t(r-F)}\]
where $C$ is the constant of integration.
Setting $|N| = N_t$, the population size at time $t$, yields
\begin{equation}
\label{eqn:NtC}
N_t=\frac{K(r-F)}{Ce^{-t(r-F)}+r}
\end{equation}
A formula suitable for computing population size at successive
time steps can be found by setting $N_t = N_{t-\Delta t}$ at time
$t=t-\Delta t$ in equation (\ref{eqn:NtC}).
The integration constant becomes
\begin{equation}
C=\Bigg(\frac{K(r-f)}{N_{t-\Delta t}}-r\Bigg)e^{(t-\Delta t)(r-F)},
\end{equation}
and finally
\begin{equation}
\label{eqn:intschaefer}
N_t = \frac{K(r-F)}{\frac{K(r-F)}{N_{t-\Delta t}}e^{-\Delta t(r-F)}-re^{-\Delta t(r-F)} -r}
\end{equation}
%\help{
Further simplification of this equation may be possible, but I have
not found it. In any case, equation (\ref{eqn:intschaefer}) is the
only general solution of the Schaefer ODE that I have seen, and it appears
to work well in numerical applications.
%}

\subsection{Two populations with exchange}
The motivation for the two popultion Schaefer model with exchange is
developed fully elsewhere. %in Appendix~\ref{sec:models}.

The basic equations can be written
\begin{eqnarray}
\label{eqn:xschaefer1}
\frac{d\None}{dt}&=&\None\Big(r-F-T_{12}-2(1-q)\frac{r}{K}\Ntwo-\frac{r}{K}\None\Big)\\
\nonumber\\
\label{eqn:xschaefer2}
\frac{d\Ntwo}{dt}&=&\Ntwo\Big(r-F-T_{12}-2q\frac{r}{K}\None-\frac{r}{K}\Ntwo\Big)+T_{21}
\end{eqnarray}
where $\None$ is the biomass of fish originating in region~1
and residing in region~1,
and $\Ntwo$ is the biomass of fish originating in region~2
but residing in region~1.
The parameters $r$, $K$ and $F$ are unchanged from
equation~(\ref{eqn:sschaefer}), and
$T_{12}$ is the emigration rate from region~1 ($t^{-1}$), 
$T_{21}$ is the rate of immigration of biomass from region~2 to
region~1 in units of biomass per time,
and $q\;(0 < q < 1)$ partitions the mortality caused by ``competition''
between the two subpopulations.
Substitute
\begin{eqnarray}
\label{eqn:Zdef}
Z_1&=&F+T_{12}+2(1-q)\frac{r}{K}\Ntwo\\
\nonumber\\
Z_2&=&F+T_{12}+2q\frac{r}{K}\None
\end{eqnarray}
into equations~(\ref{eqn:xschaefer1}) and~(\ref{eqn:xschaefer2})
respectively to produce model equations in a similar form to
equation~(\ref{eqn:sschaefer})
\begin{eqnarray}
\label{eqn:xschaeferZ1}
\frac{d\None}{dt}&=&\None(r-Z_1-\frac{r}{K}\None)\\
\nonumber\\
\label{eqn:xschaeferZ2}
\frac{d\Ntwo}{dt}&=&\Ntwo(r-Z_2-\frac{r}{K}\Ntwo)+T_{21}.
\end{eqnarray}
Equation~(\ref{eqn:xschaeferZ1}) can be integrated in the same manner as
equation~(\ref{eqn:sschaefer}) to yield
\begin{equation}
\label{eqn:N1tC}
\None_t=\frac{K(r-Z_1)}{C_1e^{-t(r-Z_1)}+r}.
\end{equation}


An equivalent integral for
equation~(\ref{eqn:xschaeferZ2}), possibly by completing the square and
solving the resulting quadratic, and a means to simultaneously solve
for $C_1$ and $C_2$ are required to achieve a complete solution to
the two population model differential equations.

%\help{All that remains is to find an equivalent integral for
%equation~(\ref{eqn:xschaeferZ2}) and a means to simultaneously solve
%for $C_1$ and $C_2$. }

%\begin{equation*}
%{{2\,\log \Big|{{2\,a\,x-\sqrt{b^2-4\,a\,c}+b}\over{2\,a\,x+
% \sqrt{b^2-4\,a\,c}+b}}\Big|}\over{\sqrt{b^2-4\,a\,c}}}+C=t
%\end{equation*}
%
%\begin{equation*}
%x=-{{\sqrt{b^2-4\,a\,c}\,\left(e^{{{\sqrt{b^2-4\,a\,c}\,t
% }\over{2}}-{{\sqrt{b^2-4\,a\,c}\,C}\over{2}}}+1\right)+b\,\left(e^{
% {{\sqrt{b^2-4\,a\,c}\,t}\over{2}}-{{\sqrt{b^2-4\,a\,c}\,C}\over{2}}}
% -1\right)}\over{a\,\left(2\,e^{{{\sqrt{b^2-4\,a\,c}\,t}\over{2}}-{{
% \sqrt{b^2-4\,a\,c}\,C}\over{2}}}-2\right)}}
%\end{equation*}
%
%
\clearpage
%\input{N1}
%\input{N2}
