%\documentclass[a4paper,KOMA,landscape,titlepage]{powersem}
\documentclass[letterpaper,KOMA,landscape,titlepage]{powersem}
\usepackage[stmo,button]{ifmslide}
\usepackage{graphicx}
%\usepackage{amsmath}
%\usepackage{amsfonts}
%\usepackage{listings}
%\usepackage[T1]{fontenc}
%\usepackage{mathptmx}
%\usepackage{charter}
%\usepackage{pictexwd}

%%%%%%%%%%%%%%%%%%%%%%%%%%%%%%%%%%%%%%%%%%%%%%%%%%%%%%%%%%%%%%%%%%%%
% Set some info on the pdf-file itself (optional)
%%%%%%%%%%%%%%%%%%%%%%%%%%%%%%%%%%%%%%%%%%%%%%%%%%%%%%%%%%%%%%%%%%%%
\hypersetup{pdfauthor={John Sibert}}
\hypersetup{pdfsubject={Assessment Model of MHI YFT Stocks}}
\hypersetup{pdftitle={Assessment model of Main Hawaiian Islands
Yellowfin Tuna Population}}
\hypersetup{pdfkeywords={yellowfin tuna, state space, abundance index, model, Hawaii}}
\hypersetup{pdfpagemode=UseThumbs}
\hypersetup{bookmarks=false}
\newcommand\doublespacing{\baselineskip=1.6\normalbaselineskip}
\newcommand\singlespacing{\baselineskip=1.0\normalbaselineskip}
\renewcommand\deg[1]{$^\circ$#1}
\newcommand\SD{SEAPODYM}
\newcommand\MFCL{MULTIFAN-CL}
\newcommand\ADMB{ADModel Builder}
\newcommand\SPC{Secretariat of the Pacific Community}
\newcommand\WCPO{Western Central Pacific Ocean}
\newcommand\SSAP{Skipjack Survey and Assessment Programme}
\newcommand\RTTP{Regional Tuna Tagging Programme}
\newcommand\PTTP{Pacific Tuna Tagging Programme}
\newcommand\FAD{fish aggregating device}
\newcommand\ADRM{advection-diffusion-reaction model}
\newcommand\help[1]{\color{red}{\it #1 }\normalcolor}
\newcommand\widebar[1]{\overline{#1}}
\newcommand\EEZ{Exclusive Economic Zone}

\newcommand\None{{N_{1,1}}}
\newcommand\Ntwo{{N_{2,1}}}
\newcommand\Nsum{{N_{1,1}+N_{2,1}}}
\newcommand\peryr{yr$^{-1}$}
\newcommand\prevN[1]{{#1_{t-\Delta t}}}
\newcommand\nextN[1]{{#1_t}}
\newcommand\MSY{\widetilde{Y}}
\newcommand\Fmsy{F_{\MSY}}
\newcommand\MSYFmsy{\MSY\;\Fmsy}
\newcommand\Bd{B_1\; d}

\begin{document}
\pageTransitionReplace
\pagecounter[on]
\slidepagestyle{empty}
\panelposition{outsidebottom}


%\freelogo(25,-13)[1.09cm] % outside bottom right of page counter
\freelogo(25,-13.8)[1.2cm] % outside bottom right of page counter

\definecolor{mytitle}{rgb}{0.0,0.4,0.5}
\definecolor{myauthor}{rgb}{0.9,0.9,0}
\definecolor{section1}{rgb}{0,0,.9}

%%%%%%%%%%%%%%%%%%%%%%%%%%%%%%%%%%%%%%%%%%%%%%%%%%%%%%%%%%%%%%%%%%%%
%\orgname{Universty of Hawaii}
%\orgname{Retirement-failure Consulting}
%\orgurl{http://admb-project.org/}

\author{\scalebox{1}[1.3]{John Sibert, Retirement-failure Consulting}} 
\title{Feasibility of developing a stock assessment model for Main
Hawaiian Islands Yellowfin Tuna Fishery\\
\vspace{4ex}
Part Deux}


\address{\href{mailto:sibert@hawaii.edu}{sibert@hawaii.edu}}

\begin{slide}
\maketitle
\end{slide}
\centerslidesfalse

\begin{slide}\section{Combined HDAR and NOAA Catch Time Series}
\begin{center}
\includegraphics[height=0.8\textheight]{./graphics/5_gear_catch_history_a.pdf}\\
\color{red}{No Recreational Data}\normalcolor
\end{center}
\end{slide}

\begin{slide}\section{WCPFC Stock Assessments}
\label{fig:MFCL2}
\begin{center}
\begin{tabular}{cc}
MFCL Region 2 & MFCL Region 4\\
\includegraphics[width=0.45\textwidth]{./graphics/annual_region_2_biomass.pdf}&
\includegraphics[width=0.45\textwidth]{./graphics/annual_region_4_biomass.pdf}\\
\end{tabular}
\end{center}
\end{slide}

\begin{slide}\section{Feasibility questions}
\begin{enumerate}
\item Can we contrive a simple model of the MHI YFT population and
fishery?
\item Can the model parameters be estimated from the data?
\item Are the model biomass estimates plausible?
\item Can the model results be used in alphabet soup?
\end{enumerate}
\end{slide}

\begin{slide}\section{Principle model assumptions}
\begin{enumerate}
\item The dynamics of the population of YFT in the MHI follows a
simple Schaefer model.
\item Fishing mortality is represented by a random walk.
\item Predicted catch by gear is the product of estimated fishing mortality
for each gear and average predicted biomass during a year.
\item Optional use of MFCL biomass estimate as index of abundance so
that local abundance is {\bfseries approximately proportional} to the
index biomass.
\end{enumerate}
\end{slide}

\begin{slide}\section{Technical features}
\begin{enumerate}
\item Fishing mortality and biomass are random effects.
\item Process errors associated with population growth, fishing
mortality random walk, and biomass index proportionality are assumed
to be equal and represented by a single parameter $(\sigma_P)$.
\item Two alternate logistic model parameterizations:
\begin{enumerate}
\item $K = \frac{4\MSY}{r};\quad r = 2\Fmsy$
\item $K = d\cdot B_1$
\end{enumerate}
\item Zero-inflated log-normal catch likelihood.
\item Optional log-normal prior on $r$ with 
$\tilde{r} = 0.486$ and $\sigma_r = 0.8$,
% (Carruthers and McAllister, 2011).
\item Analytic solution to Schaefer ODE for stable propagation
through time.
\item All computer code, data files, and draft reports in support of this
analysis can be found at Github:
\url{https://github.com/johnrsibert/XSSA.git}.
\end{enumerate}
\end{slide}
\begin{slide}\section{Estimabilty}
{\scriptsize
\label{tag:ests4}
\begin{center}
\begin{tabular}{|ll|rr|rr|}
\hline
Index && \multicolumn{2}{c|}{None}&\multicolumn{2}{c|}{MFCL 2}\\
%\cline{3-4}\cline{5-6}
\hline
\multicolumn{2}{|c|}{Parameterization}&$\MSYFmsy$&$\Bd$&$\MSYFmsy$&$\Bd$\\
\multicolumn{2}{|c|}{Designation}& A & B& C& D\\
\hline
\hline
$|G|_{max}$& Gradient at Minimum & 0.0016409 & 33.1289 & 3.51082e-05 & 3.77653\\
$n$ & Estimated Parameters &4 & 5 & 5 & 6\\
$-\log L$& Likelihood & -237.238 & -237.968 & -247.175 & -243.343\\
AIC & Akaike Criterion & -466.476 & -465.936 & -484.35 & -474.686\\
\hline
$B_1$& Initial Biomass & --- & 1184.2 & --- & 2802.3\\
$d$ &$K=dB_1$ & --- & 9.6674 & --- & 2.6348\\
$\MSY$& MSY & 1147.5 & (1199.3) & 1288.7 & (1032.6)\\
$\Fmsy$& F at MSY & 0.82239 & (0.20952) & 0.1668 & (0.2797)\\
$r$& Growth Rate & (1.6448) & 0.41904 & (0.3336) & 0.5594\\
$K$& Equilibrium Biomass & (2790.8) & (11448) & (15452) & (7383.5)\\
$\sigma_P$& Process Error & 0.37416 & 0.36757 & 0.2743 & 0.2649\\
$\sigma_Y$& Observation Error & 0.41693 & 0.43062 & 0.46924 & 0.47614\\
$Q$& Index Proportionality  & --- & --- & 0.04321 & 0.016535\\
\hline
\end{tabular}
\end{center}
}
%\renewcommand{\baselinestretch}{0.1}
%\par{\tiny
%Model estimates from four different model configurations using the
%default prior on $r$.
%Model complexity, expressed in number of parameters estimated ($n$)
%increases from left to right. Long dashes (---) indicate parameters
%not estimated. 
%%$n$ indicates the number of parameters estimated; 
%$-\log L$ is the negative log likelihood (the smaller the number, the
%better the fit);
%$|G|_{max}$ is the curvature of the likelihood at the its apparent
%minimum (values greater than $0.01$ indicate non-convergence).
%%other variables are defined in Table~\ref{tab:allvars1}.
%}
\end{slide}

\begin{slide}\section{Estimated Biomass Trends}
\label{fig:estbiomass}
\begin{center}
\includegraphics[height=0.8\textheight]{./graphics/biomass-array.pdf}
\end{center}
\end{slide}

\begin{slide}\section{Production}
\label{fig:estprod}
\begin{center}
\includegraphics[height=0.8\textheight]{./graphics/production-array.pdf}
\end{center}
\end{slide}

\begin{slide}\section{Omitting $r$ prior}
\label{tag:ests4NOprior}
{\scriptsize
\begin{center}
\begin{tabular}{|l|rr|rr|}
\hline
Index & \multicolumn{2}{c|}{None}&\multicolumn{2}{c|}{MFCL 2}\\
\cline{2-3}\cline{4-5}
Parameterization&$\MSYFmsy$&$\Bd$&$\MSYFmsy$&$\Bd$\\
Designation& A & B& C& D\\
\hline

\hline
$n$ & 4 & 5 & 5 & 6\\
$-\log L$ & -284.898 & -236.212 & -246.302 & -242.176\\
$|G|_{max}$ & 2.45563 & 151.693 & 1.24795e-05 & 39.9125\\
\hline
$B_1$ & --- & 1540.2 & --- & ---\\
$d$ & --- & 12.567 & --- & ---\\
$\MSY$ & --- & 1274.9 & 1579.3 & ---\\
$\Fmsy$ & --- & 0.13174 & 0.1293 & ---\\
$r$ & --- & 0.26347 & 0.25859 & ---\\
$K$ & --- & 19355 & 24430 & ---\\
$\sigma_P$ & --- & 0.35682 & 0.27044 & ---\\
$\sigma_Y$ & --- & 0.43481 & 0.47162 & ---\\
$Q$ & --- & --- & 0.073752 & ---\\
\hline
\end{tabular}
\end{center} }
\end{slide}

\begin{slide}\section{Alternate forcing: MFCL Region 4}
\label{fig:estr4}
\begin{center}
\begin{tabular}{lr}
\includegraphics[width=0.45\textwidth]{./graphics/r4_est_popB1.pdf}&
\includegraphics[width=0.45\textwidth]{./graphics/r4_est_FB1.pdf}\\
\end{tabular}
\end{center}
\end{slide}

\begin{slide}\section{Conclusions}
\begin{enumerate}
\item Yellowfin catch data from fleets operating in the Main Hawaiian
Islands waters are sufficiently informative to estimate relative
biomass trends.
\item An index of abundance is required to estimate absolute biomass,
but absolute estimates are sensitive to the choice of index
population.
\item Representing trends in fishing mortality as a random walk is a
convenient and effective approach to accounting for the removal of
biomass from the fish population.
\item The Bayesian prior on $r$ is difficult to assign and probably not required.
\end{enumerate}
\end{slide}

\begin{slide}\section{Next Steps?}
\begin{enumerate}
\item Technical review of model, including statistical assumptions,
and computing methods.
\item Compare these results to Catch-MSY analysis. 
\item Review previous uses of production models in tuna fisheries.
\item Test alternative biomass indices, including MHI-specific
SEAPODYM estimates.
\item Work within WCPFC assessment process to improve applicability of
WCPFC stock assessments to local requirements.
\end{enumerate}
\end{slide}

\end{document}
