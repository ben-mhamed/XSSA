\documentclass[12pt,letterpaper]{article}
\usepackage{amsmath}
\usepackage{amsfonts}
%\usepackage{color}
\usepackage[usenames,dvipsnames]{color}
\usepackage{graphicx}
\usepackage{longtable}
\usepackage{rotating}
\usepackage{verbatim}
\usepackage[pdftex,bookmarksopen]{hyperref}
\hypersetup{pdfauthor={John Sibert}}
\hypersetup{pdfsubject={Compartment model of MHI YFT}}
\hypersetup{pdftitle={Two-compartment models of Main Hawaiian Islands
Yellowfin Tuna Population}}
\hypersetup{pdfkeywords={yellowfin,state space,compartment model,Hawaii}}

\newcommand\doublespacing{\baselineskip=1.6\normalbaselineskip}
\newcommand\singlespacing{\baselineskip=1.0\normalbaselineskip}
\renewcommand\deg[1]{$^\circ$#1}
\newcommand\SD{SEAPODYM}
\newcommand\MFCL{MULTIFAN-CL}
\newcommand\ADMB{ADModel Builder}
\newcommand\SPC{Secretariat of the Pacific Community}
\newcommand\WCPO{Western Central Pacific Ocean}
\newcommand\SSAP{Skipjack Survey and Assessment Programme}
\newcommand\RTTP{Regional Tuna Tagging Programme}
\newcommand\PTTP{Pacific Tuna Tagging Programme}
\newcommand\HTTP{Hawaii Tuna Tagging Programme}
\newcommand\WCPFC{Western and Central Pacific Fisheries Commission}
\newcommand\FAD{fish aggregating device}
\newcommand\ADRM{advection-diffusion-reaction model}
\newcommand\help[1]{\color{Magenta}{\it #1 }\normalcolor}
\newcommand\widebar[1]{\overline{#1}}
\newcommand\EEZ{Exclusive Economic Zone}

\title{Preliminary Yield per Recruit Analysis of the Hawaiian
Yellowfin Tuna Fishery}

\author{
John Sibert\thanks{sibert@hawaii.edu}\\
Joint Institute of Marine and Atmospheric Research\\
University of Hawai'i at Manoa\\
Honolulu, HI  96822 U.S.A.\\[0.125in]
\date{\today}
}

\pagestyle{myheadings}
\markright{Sibert\hfil YFT Yield Per Recruit\hfil{\bf DRAFT}\hfil\today}

\begin{document}
\maketitle

\doublespacing

\section*{Introduction}
The yield per recruit (YPR) is based on the theory of exploited fish
populations developed in the 1950s
by Beverton and Holt (1957). It is a relatively simple
approach requiring only estimates of fishing mortality (F), natural
mortality (M), and rate of growth in weight. YPR has fallen into
disuse because contemporary stock assessment methods provide more useful
biomass-based
information for fisheries managers. 
Nevertheless, YPR can provide insight and guidance regarding potential
fishery management interventions.
Derivations of
YPR often center around development of formulas for calculating YPR based on
assumptions of constant F and M over the life of the exploited fish.
Estimates of age-dependent F and M are often available from
agre-structured stock
assessments or tagging experiments. These estimates can be easily
applied to computing YPR with fewer assumptions. Sparre and Venema
(1998) suggest one approach.

Estimates of F and M at age for Hawaiian yellowfin tuna are avaialble
from two sources: the 1995-2000 \HTTP\ (HTTP) (Adam et 2003) and the
2014 \MFCL\ (MFCL) stock assessment (Davies et al 2014). Here I
apply YPR analysis to evaluate potential effects of changing the
minimum size limit in the Main Hawaiian Islands (MHI) yellowfin tuna
fishery.

\section*{MFCL}
The Hawaii Exclusive Economic Zone is split between two regions 2 and 4
in the MFCL analysis (Figure~\ref{fig:mfclreg}). 
Region 4 extends from
10\deg{S} latitude to 20\deg{N} latitude and mainly comprises 
the large-scale equatorial purse seine and longline fisheries.
Region 2 extends from 20\deg{N} latitude to 50\deg{N} latitude
and comprises the Hawaii longline fishery.
The boundary between these two regions passes through the MHI. 
The yellowfin landings from the small boat fisheries in Hawaii are not
included in the data on which the MFCL assessment is based.
Small boat fisheries catch in aggregate more yellowfin than the longline
fishery (Figure~\ref{fig:fivegears})\help{or simple table}.

The 2014 MFCL yellowfin assessment includes data from 1952 through
2012. The stock is assumed to consist of 28 quarterly age classes. 
MFCL model output routinely includes estimates of mean weight at age,
natural mortality by age class, and fishing mortality by year, age
and region. 
The MFCL mortality estimates are shown in Figure~\ref{fig:mfclmf}. 
The values of M used are the ``reference case'', i.e., 
specified values because of problems reliably estimating M. 
``Natural mortality at age was recalculated for previous assessments
using an approach applied to other tunas in the WCPO and EPO.'' (Davies
et al 2014).
For the purpose of YPR analysis, I average the fishing
mortality at age for each region from 2009 through 2014 (the last 5
years or 20 quarters). 
Fishing mortality differs sharply between regions 4 and 2. In region
4, F is generally quite high at all sizes of fish with peaks near 5kg
and 30kg. The two modes are attributable to the purse seine and
longline catches.
In region 2, F is an order of magnitude lower with no clear
modes. The lack of a mode in the smaller sizes is due to the omission
of data from non longline fleets.


The YPR analysis for regions 4 and 2 are
presented in Figures~\ref{fig:r4ypr} and~\ref{fig:r2ypr} respectively.
In region 4, either increasing or decreasing the overall fishing
mortality would cause a decrease in yield per recruit for all
fisheries.
Increasing the size at first capture to around 10 kg increases the
yield to the whole fishery from near 1.5 kg/recruit to near 2.0
kg/recruit, and increase of approximately 30\%.
In region 2, the situation is quite different. 
Increasing or decreasing the total fishing morality simply increases
or decreases the yield.
Similarly, increasing the weight at first capture merely decreases the
the total yield.

\section*{HTTP}
Adam et al (2003) used a tag-attrition model to estimate
size-dependent fishing mortality, natural
mortality and migration rates between six recapture sites from the
HTTP yellowfin and bigeye tag recaptures. 
I used a simplified version of the same
model with only one recapture site to estimate size-dependent fishing
mortality and natural mortality from the HTTP yellowfin recaptures.
The number of age groups was increased in the simplified model
from three to
eight because there were fewer recapture strata.
`
The new HTTP mortality estimates are shown in Figure\ref{fig:csmmf}.
These estimates differ from the MFCL estimates in several ways. There
are fewer age classes represented because of the relatively low numbers
of longline recaptures. The HTTP mortality estimates are generally
much higher that the MFCL estimates.
The corresponding YPR analyses are shown in Figure~\ref{fig:csmmf}.
The YPR in relation to current levels of fishing mortality suggests
that increasing F would yield only modest gains in yield.
The YPR in relation to the weight at first recapture suggests that
increasing the minimum size limit would have a deleterious effect on
YPR.

\section*{Discussion}
The MFCL stock assessment is not intended to inform management of fisheries in
the North Pacific. Division of the Hawaii EEZ between two regions
makes application to problems in Hawaii difficult. Furthermore
exclusion of a significant tonnage catch of smaller fish exacerbates
the problem. Nevertheless, the situation in region 4 is instructive.
Fishing mortality on yellowfin is higher for small-sized fish than for
larger fish, as is the suspected case in the MHI. By analogy, it is
possible that increasing the age at recapture in Hawaii could increase
the total yield to the fishery. 


The general level of F in Hawaii is
considered to be much smaller than in region 4, so conclusions about
MHI fisheries based on region MFCL 4 are purely speculative.

Yield per recruit analysis only provides information about the effects
of management actions on the yield to the fishery. YPR does not
provide insight into stock conditions and does not offer any guidance
in establishing biomass-based reference points.

\help{Still working on it.}

\section*{Conclusions}\help{Working on it.}


\section*{Math Stuff}\help{Working on it.}

\singlespacing
\vspace{4ex}
\noindent {\bf Acknowledgements.}
This work was funded by the Western Pacficic Regional Fisheries
Managment Council. I thank the Council for its generous support and
Council Staff Paul Dalzell and Eric Kingma for encouraging me to
actually take on this analysis project and for their on-going
collaboration.
Thanks to Mr. David Itano for sharing insights into the small-boat
fisheries in Hawaii.
Particular thanks to Dr. M. Shiham Adam of the Maldives Marine
Research Centre for making available computer code for estimating
mortality from HTTP tagging data.
Thanks to Mr. Reginald Kokubun of the Hawaii Division of Aquatic
Resources for supplying catch report data from the HDAR commercial
fisheries data base.
Thanks to Mr. Keith Bigelow and Ms. Karen Sender of NOAA Pacific
Island Fisheries Science Center for supplying logbook reporting data and
weight-frequency data from the PIFSC data base.
Thanks also to Dr. John Hampton of the Secretariat of the Pacific
Community, Oceanic Fisheries Programme, for making available
MULTICAN-CL output files from the latest Western and Central Pacific
Fisheries Commission yellowfin tuna stock assessment, and to Mr. Nick
Davies for sharing R scripts and advice on how to decode the MFCL
output files.

\section*{References}
{\parindent=0cm \small
\everypar={\hangindent=2em \hangafter=1}\par
\doublespacing
Adam, M. S., J. Sibert, D. Itano and K. Holland. 2003. Dynamics of
bigeye (Thunnus obesus) and yellowfin tuna (T. albacares) in Hawaii's
pelagic fishery: analysis of tagging data with a bulk transfer model
incorporating size specific attrition. Fishery Bulletin 101(2):
215-228.

Beverton, R. J. H. and S. J. Holt. 1957. On the Dynamics of Exploited
Fish Populations. Fishery Investigations Series II Volume XIX,
Ministry of Agriculture, Fisheries and Food. London: Her Majesty's
Stationary Office.

Davies, N., S. Harley, J. Hampton, S. McKechnie. 2014. Stock
assessment of yellowfin tuna in the western and central pacific ocean.
WCPFC-SC10-2014/SA-WP-04.

Itano, D., K. Holland. 2000.  Movement and vulnerability of bigeye
(Thunnus obesus) and yellowfin tuna (Thunnus albacares) in relation to
FADs and natural aggregation points.Aquat. Living Resour. 13: 213-223.

Kleiber, P., J. Hampton, N. Davies, S. Hoyle, D. Fournier. 2014.
MULTIFAN-CL User’s Guide

Quinn, T. and R. Deriso. 1999. Quantitative fish dynamics. Oxford
University Press, New York.

Sparre, P. and S. Venema. 1998. Introduction to tropical fish stock
assessment. Part 1: Manual. Food and Agricultural Organization of the
United Nations. Fisheries Technical Paper 306/1.
\par}

%%%%%%%% figures begin here %%%%%%%%%%%%%%%%%%%%%%
\begin{figure}
\begin{center}
\includegraphics[height=1.0\textwidth]{./graphics/MFCLregions.png}
\caption{\label{fig:mfclreg}
Regions used in the 2014 MFCL stock assessment; from Davies et al,
2014.
}
\end{center}
\end{figure}

\begin{figure}
\begin{center}
\includegraphics[height=0.8\textheight]{./graphics/MFCL_MF.pdf}
\caption{\label{fig:mfclmf}
Natural and fishing mortality from the 2014 MFCL yellowfin stock
assessment. The upper panel (M) shows the ``reference case'' natural
mortality at age plotted against mean weight at age.
The lower two panels, F(2) and F(4), show the MFCL estimated fishing
mortality for regions 2 and 4 respectively, averaged over the period
2009 through 2014.
The red marks on the abscissa are placed at
the current lower catch weight limit and at three other
weight limits sometimes discussed; 
that is at 3, 10, 15, and 20 pounds.
}
\end{center}
\end{figure}

\begin{figure}
\begin{center}
\includegraphics[height=0.8\textheight]{./graphics/YPR_MFCL_R4.pdf}
\caption{\label{fig:r4ypr}
Yield per recruit in MFCL region 4 as a function of fishing mortality
and age at first recapture.
Panel A shows the change yield per recruit due to multiplying
the fishing mortality at all ages by constant factor ranging from 0 to
10, that is from essentially closing all fisheries to expanding all
fisheries by a factor of 10. 
The dashed vertical red line is drawn at 1, the current fishing mortality.
Panel B shows the change in yield per recruit of increasing
the minimum size limit in the fishery from 0kg to 70kg. 
The dashed vertical red line is drawn at the weight producing the
highest yield per recruit.
}
\end{center}
\end{figure}

\begin{figure}
\begin{center}
\includegraphics[height=0.8\textheight]{./graphics/YPR_MFCL_R2.pdf}
\caption{\label{fig:r2ypr}
Yield per recruit in MFCL region 2 as a function of fishing mortality
and age at first recapture. 
See caption in Figure~\ref{fig:r4ypr} for details.
}
\end{center}
\end{figure}

\begin{figure}
\begin{center}
\includegraphics[height=0.8\textheight]{./graphics/csm_MF.pdf}
\caption{\label{fig:csmypr}
Natural mortality (A) and fishing mortality (B) estimates from the 
HTTP tag recaptures.
Mortality estimates (per quarter) are shown in orange.
The green bars are the point estimates $\pm$ two standard deviations.
The light gray lines are MFCL estimates shown for comparison.
}
\end{center}
\end{figure}

\begin{figure}
\begin{center}
\includegraphics[height=0.8\textheight]{./graphics/YPR_csm.pdf}
\caption{\label{fig:csmmf}
Yield per recruit in the Main Hawaiian Islands based on HTTP mortality
estimates. 
See caption in Figure~\ref{fig:r4ypr} for details.
}
\end{center}
\end{figure}

\end{document}
