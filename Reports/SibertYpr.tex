\documentclass[12pt,letterpaper]{article}
\usepackage{amsmath}
\usepackage{amsfonts}
%\usepackage{color}
\usepackage[usenames,dvipsnames]{color}
\usepackage{graphicx}
\usepackage{longtable}
\usepackage{rotating}
\usepackage{verbatim}
\usepackage[pdftex,bookmarksopen]{hyperref}
\hypersetup{pdfauthor={John Sibert}}
\hypersetup{pdfsubject={MHI Yellowfin Yield Per Recruit}}
\hypersetup{pdftitle={ Yield per Recruit Analysis of the Hawaiian
Yellowfin Tuna Fishery}}
\hypersetup{pdfkeywords={yellowfin,yield per recruit}}

\newcommand\doublespacing{\baselineskip=1.6\normalbaselineskip}
\newcommand\singlespacing{\baselineskip=1.0\normalbaselineskip}
\renewcommand\deg[1]{$^\circ$#1}
\newcommand\SD{SEAPODYM}
\newcommand\MFCL{MULTIFAN-CL}
\newcommand\ADMB{ADModel Builder}
\newcommand\SPC{Secretariat of the Pacific Community}
\newcommand\WCPO{Western Central Pacific Ocean}
\newcommand\SSAP{Skipjack Survey and Assessment Programme}
\newcommand\RTTP{Regional Tuna Tagging Programme}
\newcommand\PTTP{Pacific Tuna Tagging Programme}
\newcommand\HTTP{Hawaii Tuna Tagging Programme}
\newcommand\WCPFC{Western and Central Pacific Fisheries Commission}
\newcommand\FAD{fish aggregating device}
\newcommand\ADRM{advection-diffusion-reaction model}
\newcommand\help[1]{\color{Magenta}{\it #1 }\normalcolor}
\newcommand\widebar[1]{\overline{#1}}
\newcommand\EEZ{Exclusive Economic Zone}

\title{
{\bf\color{red}DRAFT REPORT\normalcolor}\\
~\\
Yield per Recruit Analysis of the Hawaiian
Yellowfin Tuna Fishery}

\author{
John Sibert\thanks{sibert@hawaii.edu}\\
Joint Institute of Marine and Atmospheric Research\\
University of Hawai'i at Manoa\\
Honolulu, HI  96822 U.S.A.\\[0.125in]
\date{\today}
}

\pagestyle{myheadings}
\markright{\color{red}Sibert\hfil YFT Yield Per Recruit\hfil{\bf
DRAFT}\hfil\today}

\begin{document}
\maketitle

\doublespacing

\section*{Introduction}
The yield per recruit (YPR) is based on the theory of exploited fish
populations developed in the 1950s
by Beverton and Holt (1957). It is a relatively simple
approach requiring only estimates of fishing mortality (F), natural
mortality (M), and rate of growth in weight. YPR has fallen into
disuse because contemporary stock assessment methods provide more useful
biomass-based
information for fisheries managers. 
Nevertheless, YPR can provide insight and guidance regarding potential
fishery management interventions.
Derivations of
YPR often center around development of formulas for calculating YPR based on
assumptions of constant F and M over the life of the exploited fish.
Estimates of age-dependent F and M are often available from
age-structured stock
assessments or tagging experiments. These estimates can be easily
applied to computing YPR with fewer assumptions. Sparre and Venema
(1998) suggest one approach.

Estimates of F and M at age for Hawaiian yellowfin tuna are available
from two sources: the 1995-2000 \HTTP\ (HTTP) (Adam et 2003) and the
2014 \MFCL\ (MFCL) stock assessment (Davies et al 2014). Here I
apply YPR analysis to evaluate potential effects of changing the
minimum size limit in the Main Hawaiian Islands (MHI) yellowfin tuna
fishery.

\section*{Estimates of Mortality and YPR}
\subsubsection*{MFCL}
The Hawaii Exclusive Economic Zone is split between two regions 2 and 4
in the MFCL analysis (Figure~\ref{fig:mfclreg}). 
Region 4 extends from
10\deg{S} latitude to 20\deg{N} latitude and mainly comprises 
the large-scale equatorial purse seine and longline fisheries.
Region 2 extends from 20\deg{N} latitude to 50\deg{N} latitude
and comprises the Hawaii longline fishery.
The boundary between these two regions passes through the MHI. 
The yellowfin landings from the small boat fisheries in Hawaii are not
included in the data on which the MFCL assessment is based.
Small boat fisheries catch in aggregate more yellowfin than the longline
fishery (Figure~\ref{fig:fivegears})\help{or simple table}.

The 2014 MFCL yellowfin assessment includes data from 1952 through
2012. The stock is assumed to consist of 28 quarterly age classes. 
MFCL model output routinely includes estimates of mean weight at age,
natural mortality by age class, and fishing mortality by year, age
and region. 
The MFCL mortality estimates are shown in Figure~\ref{fig:mfclmf}. 
The values of M used are the ``reference case'', i.e., 
specified values because of problems reliably estimating M. 
``Natural mortality at age was recalculated for previous assessments
using an approach applied to other tunas in the WCPO and EPO.'' (Davies
et al 2014).
For the purpose of YPR analysis, I average the fishing
mortality at age for each region from 2009 through 2014 (the last 5
years or 20 quarters). 
Fishing mortality differs sharply between regions 4 and 2. In region
4, F is generally quite high at all sizes of fish with peaks near 5kg
and 30kg. The two modes are attributable to the purse seine and
longline catches.
In region 2, F is an order of magnitude lower with no clear
modes. The lack of a mode in the smaller sizes is due to the omission
of data from non longline fleets.

The YPR analysis for regions 4 and 2 are
presented in Figures~\ref{fig:r4ypr} and~\ref{fig:r2ypr} respectively.
In region 4, either increasing or decreasing the overall fishing
mortality would cause a decrease in yield per recruit for all
fisheries.
Increasing the size at first capture to around 10 kg increases the
yield to the whole fishery from near 1.5 kg/recruit to near 2.0
kg/recruit, and increase of approximately 30\%.
In region 2, the situation is quite different. 
Increasing or decreasing the total fishing morality simply increases
or decreases the yield.
Similarly, increasing the weight at first capture merely decreases the
the total yield.

\subsubsection*{HTTP}
Adam et al (2003) used a size- and spatially-structured
tag-attrition model to estimate
size-dependent fishing mortality, natural
mortality and migration rates between release and recapture sites from
HTTP yellowfin and bigeye tag recaptures. 
I used a simplified, spatially aggregated, version of the same
model to estimate size-dependent fishing
mortality and natural mortality from the HTTP yellowfin recaptures.
The number of age groups was increased in the simplified model
from three to
eight because there were fewer recapture strata.

The new HTTP mortality estimates are shown in Figure~\ref{fig:csmmf}.
These estimates differ from the MFCL estimates in several ways. The
size range is more restricted because of the relatively few
recaptures of large fish from the longline. HTTP mortality estimates
are generally much higher than MFCL estimates.
The corresponding YPR analyses are shown in Figure~\ref{fig:yprcsm}.
The YPR in relation to current levels of fishing mortality suggests
that increasing F would yield only modest gains in yield.
The YPR in relation to the weight at first capture suggests that
increasing the minimum size limit would have a deleterious effect
total yield to the fishery.


\section*{Discussion}
The MFCL stock assessment is intended to inform management of
large-scale purse seine and longline fisheries in equatorial Pacific.
Application of MFCL stock assessment results to a relatively insular
small scale fishery in the North Pacific may be stretching the
capabilities of MFCL. Division of the Hawaii EEZ between two regions
makes application to problems in Hawaii difficult. Furthermore,
exclusion of the substantial catch of smaller fish from the analysis exacerbates
the problem of applying the regional stock assessment to
resolve fishery management issues in Hawaii. The results in region 4
is instructive nevertheless.
Fishing mortality on yellowfin is higher for small-sized fish than for
larger fish, as is the suspected case in the MHI. By analogy, it is
possible that increasing the age at first capture in Hawaii could increase
the total yield to the fishery. 

The high mortality estimates from the HTTP tag recaptures are
troubling. To some extend high mortality estimates reflect the
lack of precision in the definition of M. ``Natural mortality'' is not
a process that is ever observed in a way that can be easily tied to
specific parameters in a statistical model. M is the aggregate loss of fish from
the population  that cannot be attributed in some way to a
fishery. For widely distributed, highly mobile fish such as tunas,
emigration is an important component of M in addition to the
biological processes usually associated with mortality, e.g. predation and
senescence. Both the MFCL stock assessment and the previous HTTP
analysis by Adam et al (2003) account movement, so the new
spatially aggregated mortality estimates include emigration and could
thus be higher.
Although the new estimates are difficult to compare directly to the
previous HTTP estimates because they use different size
classes, the new estimates are generally slightly higher or similar to
the previous estimates. \help{It should be possible to do a better
comparison between old and new HTTP estimates.}
The YPR analysis based on HTTP mortality estimates shows that
increasing the lower size limit on yellowfin could have a deleterious
effect on the fishery.

Yield per recruit analysis only provides information about the effects
of management actions on the yield to the fishery. YPR does not
provide insight into stock conditions and does not offer any guidance
in establishing biomass-based reference points. Therefore stock conservation
decisions should not be based solely on YPR analysis.

\section*{Conclusions and Recommendations}
\begin{enumerate}
\item The YPR analysis for MFCL Region 4 shows clearly that reducing
the size at capture would increase the yield to the entire fishery.
Whether change in minimum size at recatpure would benefit the MHI
yellowfin fishery is not clear.

The US Delegation to the WCPFC should strongly advocate for minimum
size restrictions in MFCL Region 4, and elsewhere in the equatorial
fishery.

\item The YPR analysis for MFCL Region 2 is ambiguous because only
longline catches are included in the MFCL assessment and because the
MFCL regions are ill-addapted to support management of fisheries in
Hawaii.

US scientists should participate actively in the WCPFC pre-assessment
workshop to assist the assessment team to find ways to include more
data from Hawaii fisheries in the assessment and to redefine the MFCL
regions to be more informative about the Hawaiian yellowfin
population.

\item The YPR analysis using mortality estimates from tagging data are
are inconclusive, but there is no clear benefit to the fishery of
increasing the minimum size restrictions.

The HTTP data should be completely reanalyzed, and if any yellowfin
recaptures have been return since 2001, the new recaptures should be
included in the data.
\end{enumerate}


\section*{Math Stuff}
Yield per recruit is an estimate of the contribution to the fishery of
a cohort during its entire life span.
The biomass of fish of age $a$ is the product of the number of fish
of age $a$ in the population $(N_a)$ times the weight of age $a$ fish
$(W_a)$ that is $B_a=N_a\cdot W_a$, and the contribution of age $a$
fish to the yield $Y_a=F_a\cdot B_a = F_a\cdot N_a\cdot W_a$, 
where $F_a$ is the fishing mortality at age.  
The yield per recruit is thus the $\sum_a Y_a/R$, where $R$ is the
recruitment at age $a=0$.
%The change in biomass with age is
%thus,\[ \frac{dB}{da} = N\frac{dW}{da}+W\frac{dN}{da} \].
%This expression can be approximated by finite differences as
%\[
%\frac{dB}{da} \approx
%\frac{B_a-B_{a-\Delta a}}{\Delta a} = 
%   N_{a-\Delta a}\frac{W_a-W_{a-\Delta a}}{\Delta a} + 
%   W_{a-\Delta a}\frac{N_a-N_{a-\Delta a}}{\Delta a}
%\].
%
Assuming that $R = 1 = N_0$, yield per recruit can be easily computed by these
relatively simple relations
\begin{eqnarray}
Z_a &=& M_a +F_a \\
N_a &=& N_{a-\Delta a}e^{-Z_a\Delta a}\\
Y_a &=& F_aN_aW_a.\\
\end{eqnarray}
$F_a$ and $M_a$ are estimated by the MFCL assessment or the HTTP
tag attrition models.
$W_a$ can be computed from estimated or specified
growth parameters and length-weight relationships. In the case of the
MFCL assessment, $W_a$ is a routine model output. In the case of the
HTTP analysis, $W_a$ is computed by von Bertalanffy growth parameters
and the length-weight relationship from the MFCL assessment.

All computer code, data files, and draft reports in support of this
analysis can be found at Github
\url{https://github.com/johnrsibert/XSSA.git}.

\singlespacing
\vspace{4ex}
\noindent {\bf Acknowledgements.}
This work was funded by the Western Pacficic Regional Fisheries
Managment Council. I thank the Council for its generous support and
Council Staff Paul Dalzell and Eric Kingma for encouraging me to
actually take on this analysis project and for their on-going
collaboration.
Thanks to Mr. David Itano for sharing insights into the small-boat
fisheries in Hawaii.
Particular thanks to Dr. M. Shiham Adam of the Maldives Marine
Research Centre for making available computer code for estimating
mortality from HTTP tagging data.
Thanks to Mr. Reginald Kokubun of the Hawaii Division of Aquatic
Resources for supplying catch report data from the HDAR commercial
fisheries data base.
Thanks to Mr. Keith Bigelow and Ms. Karen Sender of NOAA Pacific
Island Fisheries Science Center for supplying logbook reporting data and
weight-frequency data from the PIFSC data base.
Thanks also to Dr. John Hampton of the Secretariat of the Pacific
Community, Oceanic Fisheries Programme, for making available
MULTICAN-CL output files from the latest Western and Central Pacific
Fisheries Commission yellowfin tuna stock assessment, and to Mr. Nick
Davies for sharing R scripts and advice on how to decode the MFCL
output files.

\section*{References}
{\parindent=0cm \small
\everypar={\hangindent=2em \hangafter=1}\par
Adam, M. S., J. Sibert, D. Itano and K. Holland. 2003. Dynamics of
bigeye (Thunnus obesus) and yellowfin tuna (T. albacares) in Hawaii's
pelagic fishery: analysis of tagging data with a bulk transfer model
incorporating size specific attrition. Fishery Bulletin 101(2):
215-228.

Beverton, R. J. H. and S. J. Holt. 1957. On the Dynamics of Exploited
Fish Populations. Fishery Investigations Series II Volume XIX,
Ministry of Agriculture, Fisheries and Food. London: Her Majesty's
Stationary Office.

Davies, N., S. Harley, J. Hampton, S. McKechnie. 2014. Stock
assessment of yellowfin tuna in the western and central pacific ocean.
WCPFC-SC10-2014/SA-WP-04.

Itano, D., K. Holland. 2000.  Movement and vulnerability of bigeye
(Thunnus obesus) and yellowfin tuna (Thunnus albacares) in relation to
FADs and natural aggregation points.Aquat. Living Resour. 13: 213-223.

Kleiber, P., J. Hampton, N. Davies, S. Hoyle, D. Fournier. 2014.
MULTIFAN-CL User’s Guide

Quinn, T. and R. Deriso. 1999. Quantitative fish dynamics. Oxford
University Press, New York.

Sparre, P. and S. Venema. 1998. Introduction to tropical fish stock
assessment. Part 1: Manual. Food and Agricultural Organization of the
United Nations. Fisheries Technical Paper 306/1.
\par}

%%%%%%%% figures begin here %%%%%%%%%%%%%%%%%%%%%%
\begin{figure}
\begin{center}
\includegraphics[height=1.0\textwidth]{./graphics/MFCLregions.png}
\caption{\label{fig:mfclreg}
Regions used in the 2014 MFCL stock assessment; from Davies et al,
2014.
}
\end{center}
\end{figure}

\begin{figure}
\begin{center}
\includegraphics[height=0.8\textheight]{./graphics/MFCL_MF.pdf}
\caption{\label{fig:mfclmf}
Natural and fishing mortality from the 2014 MFCL yellowfin stock
assessment. The upper panel (M) shows the ``reference case'' natural
mortality at age plotted against mean weight at age.
The lower two panels, F(2) and F(4), show the MFCL estimated fishing
mortality for regions 2 and 4 respectively, averaged over the period
2009 through 2014.
The red marks on the abscissa are placed at
the current lower catch weight limit and at three other
weight limits sometimes discussed; 
that is at 3, 10, 15, and 20 pounds.
}
\end{center}
\end{figure}

\begin{figure}
\begin{center}
\includegraphics[height=0.8\textheight]{./graphics/YPR_MFCL_R4.pdf}
\caption{\label{fig:r4ypr}
Yield per recruit in MFCL region 4 as a function of fishing mortality
and age at first capture.
Panel A shows the change yield per recruit due to multiplying
the fishing mortality at all ages by constant factor ranging from 0 to
10, that is from essentially closing all fisheries to expanding all
fisheries by a factor of 10. 
The dashed vertical red line is drawn at 1, the current fishing mortality.
Panel B shows the change in yield per recruit of increasing
the minimum size limit in the fishery from 0kg to 70kg. 
The dashed vertical red line is drawn at the weight producing the
highest yield per recruit.
}
\end{center}
\end{figure}

\begin{figure}
\begin{center}
\includegraphics[height=0.8\textheight]{./graphics/YPR_MFCL_R2.pdf}
\caption{\label{fig:r2ypr}
Yield per recruit in MFCL region 2 as a function of fishing mortality
and age at first capture. 
See caption in Figure~\ref{fig:r4ypr} for details.
}
\end{center}
\end{figure}

\begin{figure}
\begin{center}
\includegraphics[height=0.8\textheight]{./graphics/csm_MF.pdf}
\caption{\label{fig:csmmf}
Natural mortality (A) and fishing mortality (B) estimates from the 
HTTP tag recaptures.
The blue bars are the point estimates $\pm$ two standard deviations.
The light gray lines are MFCL estimates shown for comparison.
}
\end{center}
\end{figure}

\begin{figure}
\begin{center}
\includegraphics[height=0.8\textheight]{./graphics/YPR_csm.pdf}
\caption{\label{fig:yprcsm}
Yield per recruit in the Main Hawaiian Islands based on HTTP mortality
estimates. 
See caption in Figure~\ref{fig:r4ypr} for details.
}
\end{center}
\end{figure}

\end{document}
